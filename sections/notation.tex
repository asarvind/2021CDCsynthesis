We denote real numbers by $\reals$ and complex numbers by $\cnums$.  For a real number subset $S\subseteq \reals$, $r\in\reals$ and $\bowtie\in\set{\leq, <,\geq,>}$, we denote $S_{\bowtie r} = \set{x\in S\st{x\bowtie r}}$.  If $X,Y\in\reals^{n\times m}$ are two real matrices of same size, then we say $X\bowtie Y$ if $\forall i,j~X_{ij}\bowtie Y_ij$.  Similarly, for a scalar $r\in\reals$, $X\bowtie r$ if $\forall i,j~X_{ij}\bowtie r$.  For a complex number $z = x+\iota y$, $\abs{z} = \sqrt{x^2+y^2}$.  For a complex matrix $X\in\cnums^{n\times m}$, $\abs{X}$ is a real matrix where $\forall i,j~\abs{X}_{ij} = \abs{X_{ij}}$.  The euclidean norm of a complex vector $x\in\cnums^n$ is $\norm{x}$ and that of a complex matrix $X\in\cnums^{n\times m}$ is $\norm{X}$.  The real valued ball of radius $r$ around a real vector $x\in\reals^n$ is $\ball{x}{r} = \set{y\in\reals^n\st{\norm{x-y}}\leq r}$.  The real projection of a set $S\subseteq \cnums^n$ is $\real{S}$.  The transpose of a matrix $P$ is denoted $P^T$ and its inverse it $P^{-1}$.  The pseudoinverse of a complex matrix $P$ where $PP^T$ is invertible is $\pinv{P} = P^T\rb{PP^T}^{-1}$.
%The Minkowski sum of two subsets $\Psi_1$ and $\Psi_2$ of a complex
%or real valued vector space with same dimension is
%$\Psi_1\bigoplus \Psi_2 = \set{x+y\st{x\in\Psi_1,~y\in\Psi_2}}$.
