We consider a sampled data linear time invariant (LTI) system  defined as follows. The state of the system at time $t\in[0,\infty)$ is $\state{t}\in\reals^n$, the feedback input  is $\inp{t}\in\reals^m$, $\lb$ and $\ub$ are lower and upper bounds, respectively, on feedback sampling period, $A_s\in\reals^{n\times n}$, $B_s\in\reals^{n\times m}$, $A_u\in\reals^{m\times n}$, $B_u\in\reals^{m\times m}$, $F_p\in\reals^{m\times (n+m)}$ and $F_c\in\reals^{m\times n}$ are real matrices related to the system dynamics described below. The matrix $F_c$ acts on the current sample and $F_p$ acts on previous sample and input.  The value $\theta$ is the time elapse after $\lb$ at which $F_p$ is applied on the previous sample to reset the feedback input.  But we apply the feedback  at most once between successive state sampling.  Allowing multiple such historical feedbacks between successive state samples makes the stabilization problem more difficult, which is beyond the scope of this paper.  Henceforth, we take $\theta\geq\rb{\ub+\lb}/2$ so that the feedback based on previous sample is applied at most once between successive state sampling.
%
\begin{align*}
    & \theta \geq \rb{\ub+\lb}/2\\
    & \exists \rb{t_k}_{k=0}^\infty,~\forall k\in\integers_{\geq 0} ~t_k\in\reals_{\geq 0}\\
    & \forall t\notin\bigcup_{k=0}^\infty\rb{ \set{t_k}\bigcup \set{t_k+\theta\st{ t_k+\theta< t_{k+1} } } },\\
    &\der{\state{t}} = A_s\state{t} + B_s{\inp{t}},~~\der{{\inp{t}}} = A_u\state{t} + B_u{\inp{t}}~\numberthis\label{eqn:ncsstart}\\
    & \forall k\in\integers_{\geq 0},
     \rb{t_{k+1}-t_k}\in[\lb,\ub],~t_0 = 0,\\
    & \inp{t_k} = F_c\state{t_k}^-~\numberthis\\
    &~\text{If}~\theta<\rb{t_{k+1}-t_k},~\text{then}~\inp{\rb{t_k+\theta}} = F_p\mymatrix{\state{t_k}^-\\\inp{t_k}^-}~\numberthis\label{eqn:ncsend}.
\end{align*}
%
We denote the combined vector of the state of the system and controller input, called {\em combined state}, at any time $t$ as $\cs{t} = \mymatrix{\state{t}^T,\inp{t}^T}^T$.  We call a sequence of combined states at sampling times $\rb{\cs{t_k}}_{k=0}^\infty$ a sampling time trajectory which, by simple manipulation of the above equations, can be shown to be equivalently governed by the following dynamics:
%
\begin{align*}
    & A_r = \mymatrix{\identity{n} & 0\\ F_c & 0},~~H = \mymatrix{\identity{n} ~~~~ 0\\ F_p\exp\rb{-A_c\rb{\theta+\lb}} },\\
    & A_c = \mymatrix{A_s & B_s \\ A_u & B_u}.~\numberthis\label{eqn:impulse}\\
    & \cs{t_{k+1}} = \R{\rb{t_{k+1}-t_k}}\cs{t_k}~~\text{where}\\
    & \forall \tau\in[\lb,\lb+\theta],~\R{\tau} = A_r\exp\rb{A_c\tau}.~\numberthis\label{eqn:action1}\\
    & \forall \tau\in(\lb+\theta,\ub],~\text{as}~\theta\geq\rb{\ub+\lb}/2, \\
    & \R{\tau} = A_r\exp\rb{A_c\rb{\tau-\lb-\theta}}H\exp\rb{A_c\rb{\lb+\theta}}.~\numberthis\label{eqn:action2}
\end{align*}
%
\begin{definition}[Exponential stability]
The system is said to be $\lambda$-exponentially stable for $\lambda\in[0,1)$ if there exists $r\in\reals_{\geq 0}$ such that for every sequence of sampling times $\rb{t_k}_{k=0}^\infty$, the following is true.
%
\begin{align*}
    & \norm{\cs{t_k}}\leq r\lambda^k\norm{\cs{0}} ~\numberthis\label{eqn:expstability}
\end{align*}
%
\end{definition}
%
\begin{problem}
We are given $\lambda\in[0,1)$, $\lb$, $\ub$, $A_s$, $B_s$, $A_u$, $B_u$ and $F_c$ such that the matrix $\R{\lb}$ is Schur.  Find a feedback matrix  $F_p\in\reals^{m\times\rb{n+m}}$ acting on the previous state sample and input and also find a $\theta\in\sqb{\rb{\ub+\lb}/2, \ub}$ such that the system is $\lambda$-exponentially stable.
\end{problem}
%
\begin{remark}~\label{rem:counterintuitive}  Let us consider that a system is stable for sampling times between $\sqb{\lb,\lb+\mu}$ without the previous sample based feedback.  The expected sampling time feedback at a time $\rb{t_k+\mu}$ is ${F_c\state{\rb{t_k+\mu}}}$.  As a function of the previous sample $\state{t_k}$, this feedback can be written as $F_c\exp\rb{A_c\rb{\lb+\mu}}\state{t_k}$. Therefore, when the sample information is missing at $\rb{t_k+\lb+\mu}$, it may seem intuitive to apply feedback $F_c\exp\rb{A_c\rb{\lb+\mu}}\state{t_k}$ at $\rb{\lb+\mu}$ to stabilize the system for a larger $\ub>\lb+\mu$.   However, counterintuitively, this may not provide stabilization for a large enough $\ub$ but other values of $F_p$ may lead to far more robust stabilization, as illustrated in the below example.  
\end{remark}
%
\begin{example}~\label{eg:counterintuitive}
Let consider the benchmarked sampled data system taken from~\cite{hetel2013stabilization,al2015stability}, where $A_s = \mymatrix{0 & 1\\ -2 & 0.1}$, $B = \mymatrix{0 \\ 1}$, $A_u = \mymatrix{0 & 0}$, $B_u = 0$ and $F_c = \mymatrix{1 & 0}$.  
%
Without previous sample based feedback,~\cite{al2015stability} has
shown this system is stable for sampling times in $\sqb{0.4,1.888}$,
but unstable for $\ub\geq 1.889$.  If we set a matching feedback
matrix $F_p = F_c\exp\rb{1.888A_c}$ acting on previous sample at
sampling time $1.888$ where $\theta = 1.488$, the system will be
unstable for $\ub\geq 1.898$, which is a very small increment $(0.01)$ than the previous stability threshold $1.888$.  This is because $\R{0.4}\R{1.898} =
A_r\exp\rb{0.4A_c}A_r\exp\rb{0.01 A_c}A_r\exp\rb{1.888 A_c}$ has
eigenvalues outside the unit complex circle.  Instead, if we take $F_p
= \mymatrix{0.1422 & 0.4669 & 0.1417}$ with $\theta = 0.41$, we can
stabilize the system for a far larger value $\ub = 3.42$, which we
shall deduce later in this paper.  This shows that the matching time
evolution based feedback $F_p = F_c\exp\rb{A_c\rb{\lb+\mu}}$ at some
$\rb{\lb+\mu}$ need not be an efficient solution.  Moreover, there is
no guarantee that such feedback will stabilize the system for a larger
$\ub>\lb+\mu$.
\end{example}

Alternatively, we shall describe a stabilization method using complex zonotopes, which stabilizes the system for large uncertainties in the sampling period on benchmark examples.
