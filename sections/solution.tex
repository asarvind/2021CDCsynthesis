%\subsection{Relation between invariant set and stability}
The stability of the system can be established the existence of a contractive set around origin containing the ball of unit radius, as described in the following lemma.
%
\begin{lemma}~\label{lem:invstability}
The system is $\lambda$-exponentially stable if there exists a set $\Psi\subseteq\reals$ such that $\max_{z\in\Psi}\norm{z}<\infty$, $\ball{0}{1}\subseteq \Psi$ and $\forall \tau\in\sqb{\lb,\ub}$ $\R{\tau}\Psi\subseteq \lambda\Psi$.
\end{lemma}
%
\begin{proof}
Let us denote $r = \max_{z\in\Psi}\norm{z}$ which is finite as considered in the above lemma.  We have $\cs{0}\subseteq \norm{\cs{0}}\ball{0}{1}\subseteq\norm{\cs{0}}\Psi$.  Then we derive the following.
%
\begin{align*}
    & \cs{k} = \prod_{i=1}^k\R{\rb{t_k-t_{k-1}}}\cs{i} \subseteq \prod_{i=1}^k\R{\rb{t_k-t_{k-1}}}\norm{\cs{0}}\Psi\\
    & = \norm{\cs{0}}\prod_{i=1}^k\R{\rb{t_k-t_{k-1}}}\Psi \subseteq \norm{\cs{0}}\lambda^k\Psi
\end{align*}
%
Then we get $\norm{\cs{k}}\leq \norm{\cs{0}}\lambda^k\max_{z\in\Psi}\norm{z} = r\norm{\cs{0}}\lambda^k$.  As this $r$ is constant for all initial states $\cs{0}$, we get that the system is $\lambda$-exponentially stable.
\end{proof}
%
Therefore, to stabilize the system, we can find an $F_p$, $\theta$ and a set $\Psi$ satisfying the conditions of Lemma~\ref{lem:invstability}.  The choice of the set $\Psi$ affects the robustness of our solution, i.e., the $\ub$ for which stabilization can be achieved.  It has been previously shown that using the eigenstructure of the system dynamics inside a set representation called complex zonotope~\cite{adimoolam2016using}, stability can be verified for larger ranges of $\ub$ on benchmark examples than other set methods.  So, in this paper, our choice for the $\Psi$ is a complex zonotope and we extend its application to stabilize the system, i.e. find $F_p$ and $\theta$.  Firstly, we discuss necessary operations on a complex zonotope in the following section.
%
\subsection{Complex Zonotope and Set Operations}
A simple zonotope is a Minkowski sum of line segments in the real euclidean space described as linear combination of real-valued vectors, called generators, with bounded combining coefficients.  However, the eigenstructure of the dynamics, which is complex valued, is closely related to finding invariants or contractive sets (see below Theorem~\ref{thm:eig}).  Since real zonotopes can not incorporate complex valued eigenvectors as generators, we extended real zonotopes to complex zonotopes in~\cite{adimoolam2016using,adimoolam2016template,adimoolam2018calculus}.  A complex zonotope is a linear combination of complex-valued vectors with complex combining coefficients bounded inside circles of complex plane.  Complex zonotopes are geometrically more expressive as they can be Minkowski sum of some embedded ellipses in addition to line segments.   They are defined below.
%
\begin{definition}
Let ${ P\in\cnums^{\rows{P}\times\cols{P}} }$ be a complex matrix, ${c\in\reals^{\rows{P}}}$ be a real vector and $s\in\reals^{\cols{P}}_{\geq 0}$ be a non-negative real vector. The following is a complex zonotope centered at $c$ with template $P$ and scale vector $s$.%
\begin{align*}
    & \cz{P}{c}{s} = \set{P\zeta + c\st{\zeta\in\cnums^{\cols{P}},~\abs{\zeta}\leq s}}
\end{align*}
%
\end{definition}
%
The following theorem illustrates the relation between eigenstructure of the system and finding contractive complex zonotopes.
%
\begin{theorem}~\label{thm:eig}
Let a complex matrix $P$ contain the eigenvectors of $\R{t}$ for some $t$.  If the eigenvalues of $\R{t}$ are all less than $\lambda$, then $\R{t}\cz{P}{0}{s}\subseteq \lambda\cz{P}{0}{s}$ for all $s\in\reals_{\geq 0}$.
\end{theorem}
%
\begin{proof}
The theorem is proved in~\cite{adimoolam2018calculus} (Lemma~5.4.1).
\end{proof}
%
The linear transformation of a complex zonotope gives another complex zonotope, which can also be computed very efficiently as follows.
%
\begin{lemma}~\label{lem:linsum}~\cite{adimoolam2018calculus}
Let us consider a complex zonotope $\cz{P}{c}{s}$ and a real matrix $A$.  Then the following is true.
%
\begin{align*}
    & A\cz{P}{c}{s} = 
    \cz{AP}{Ac}{s}.~\numberthis\label{eqn:minsum}
\end{align*}
%
\end{lemma}
%
\begin{proof}
The lemma is proved in~\cite{adimoolam2018calculus}.
\end{proof}
%
The algorithm we propose later to stabilize the system involves checking inclusion of a set of complex zonotopes, obtained by applying a sequence of transformations, inside another original complex zonotope (see Lemma~\ref{lem:invstability}).  In this regard, we define a set of complex zonotopes whose templates lie in the neighborhood of a given template as follows.  Let us consider a real matrix with positive entries $\Upsilon\in\reals_{\geq 0}^{\rows{Q}\times\cols{Q}}$ where $Q$ is a complex matrix and a real vector $\rho\in\reals^{\cols{e}}$ where $e$ is a real vector.  
%
\begin{align*}
    & \cz{Q\errplus\Upsilon}{e\errplus\rho}{r} \\
    & = \set{\cz{Q+\widehat{Q}}{e+u}{r}~\st{\abs{Q}\leq \Upsilon,~\abs{u}\leq \rho}}
\end{align*}
%
The following relation is a sufficient condition for checking the required inclusion which is proved later in Lemma~\ref{lem:inclusion}.  The following Lemma~\ref{lem:inclusion} is an extension of the inclusion checking between two complex zonotopes proved in~\cite{adimoolam2018calculus} (Theorem 2.3.8 ).
%
\begin{definition}~\label{defn:inc}  Let $P$ be a complex matrix such that $P^T P$ is a square invertible matrix.  Let $\Upsilon>0$ be a real matrix with only positive elements.  We define the relation \[ \cz{Q\errplus \Upsilon}{e\errplus\rho}{r} \order \cz{P}{c}{s}\] if all of the following conditions are verified:
%
\begin{align*}
    & \exists X,\Delta\in\cnums^{\cols{P}\times\cols{Q}}, y\in\cnums^{\cols{P}}:\\
    & PX = Q\diag{r},~~ \Delta = \abs{\pinv{P}}\Upsilon\diag{r},~\numberthis\label{eqn:inc1}\\
    & \left(e-c\right) = Py,~~\delta = \abs{\pinv{P}}{\rho},\numberthis\\
    & \forall i\in\set{1,\ldots,\rows{X}}\\
    & \abs{y_i} + \delta_i + \sum_{j=1}^{\cols{X}}\abs{X_{ij}} + \Delta_{ij}\leq s_i~\numberthis\label{eqn:inc2}
\end{align*}
\end{definition}
%
\begin{lemma}~\label{lem:inclusion}
If $ \cz{Q\errplus\Upsilon}{e\errplus\rho}{r}\order \cz{P}{c}{s}$ is true for $\Upsilon,\rho>0$, then the subset inclusion $ \cz{Q\errplus\Upsilon}{e\errplus\rho}{r}\subseteq \cz{P}{c}{s}$ is true. 
\end{lemma}
%
\begin{proof}
Let us assume that  $ \cz{Q\errplus \Upsilon}{e\errplus \rho}{r}\order \cz{P}{c}{s}$ is true.  Hence, there exist matrices $X,\Delta$ and complex vectors $y,\rho$ such that all the equations in~(\ref{eqn:inc2}) are true.
%
% \begin{align*}
% & PX = Q\diag{r},~~\left(e-c\right) = Py\\
% & \forall i\in\set{1,\ldots,\rows{X}} ~~\abs{y_i} + \sum_{j=1}^{\cols{X}}\abs{X_{ij}} \leq s_i.~\numberthis\label{proof:inclusion}
% \end{align*}
%
Let us consider any $x\in\cz{Q\errplus \Upsilon}{e}{R}$.  Based on the definition of a complex zonotope, there exists $\zeta\in\cnums^{\cols{P}}$ such that $\abs{\zeta}\leq s$ and $x = \rb{Q+\widehat{Q}}\zeta + e + u$, $\abs{\widehat{Q}}\leq \Upsilon$ and $\abs{u}\leq \rho$.  We now have to show that $x\in \cz{P}{c}{s}$.  

Let us consider a vector $\alpha\in\cnums^{\cols{\zeta}}$  such that for any $i\in\set{1,\ldots,\cols{\zeta}}$, the following is true:
%
\begin{align*}
    & \alpha_i = \zeta_i/r_i ~~\text{if}~~r_i>0,
    & \alpha_i = 0~~\text{if}~~r_i = 0~\numberthis
\end{align*}
%
Since $\abs{\zeta}\leq r$, it follows from the above definition that $\abs{\alpha}\leq 1$ and  $\zeta = \diag{r}\alpha$. 
Then we derive the following.
%
\begin{align*}
    & x = \rb{Q+\widehat{Q}}\zeta + e + u = \rb{Q+\widehat{Q}}\diag{r}\alpha + e + u \\
    & = \rb{Q+\widehat{Q}}\diag{r}\alpha + (e-c) + c + u
\end{align*}
%
and from \eqref{eqn:inc1}
\begin{align*}
    & x = P\left(X\alpha + \pinv{P}\widehat{Q}\diag{r}\alpha + y + \pinv{P}u\right) + c~\numberthis\label{proof:inc:exp}
\end{align*}
We derive the following for any $i\in\set{1,\ldots, \rows{X}}$
%
\begin{align*}
 &   \abs{X\alpha + \pinv{P}\widehat{Q}\diag{r}\alpha + y +\pinv{P}u}_i\\
 & \leq \abs{y}_i + \abs{\pinv{P}}\abs{u}  \\
 & ~~~~+ \sum_{j=1}^{\cols{X}} \rb{\abs{X}_{ij}+\rb{\abs{\pinv{P}}\abs{\widehat{Q}}\diag{r}}_{ij}}\abs{\alpha}_j\\
 & \leq \abs{y}_i + \delta_i + \sum_{j=1}^{\cols{X}}\abs{X}_{ij}+\Delta_{ij}\\
     & \leq s_i.~\numberthis\label{proof:inc:bound}
\end{align*}
%
The above second inequality is true because $\abs{\alpha}\leq 1$, $\abs{\widehat{Q}}\leq \Upsilon$, $\Delta = \abs{\pinv{P}}\Upsilon\diag{r}$, $\abs{u}\leq\rho$ and $\pinv{P}\rho = \delta$, and the last inequality is deduced from \eqref{eqn:inc2}.
%
From \eqref{proof:inc:exp} and \eqref{proof:inc:bound}, we get that $x\in\cz{P}{c}{s}$.  As this is true for any $x\in\cz{Q}{e}{r}$, the inclusion $\cz{Q}{e}{r}\subseteq \cz{P}{c}{s}$ is true.  
\end{proof}
%
\begin{remark}~\label{rem:convex}
For fixed $P$, $Q$ and $\Upsilon$, the Equations~\ref{eqn:inc1}-\ref{eqn:inc2} define convex constraints on $s$.  In fact, the inequalities can be recast as a set of second order conic constraints on $s$ and the remaining variables.  
Similarly, for fixed $s$, the equations define convex constraints on $P$, $Q$, and $\Upsilon$.
\end{remark}
%
\subsection{Stabilization using complex zonotope}
For deriving the sufficient condition to establish stabilization in Theorem~\ref{thm:main}, we use a bound on $\R{t+\delta}$ for $0\leq \delta\leq \epsilon$ state in the following lemma.
%
\begin{lemma}[~\cite{hetel2013stabilization}~]~\label{lem:convhull}
Let $L_0,L_1,...,L_r$ be a finite sequence of real matrices and $U_j\rb{\delta}= \sum_{i=1}^j L_i\delta^i$.  If $0\leq \delta<\epsilon$, then $U_r\rb{\delta} \in \conv{U_{0}\rb{\epsilon},\ldots,U_r\rb{\epsilon}}$.
\end{lemma}
\begin{proof}
This lemma is proved in~\cite{hetel2007lmi}.
\end{proof}
%
The following theorem contains the sufficient conditions for stabilization of the system, which are derived based on the   operations on a complex zonotope described earlier and the above Lemma~\ref{lem:convhull}.  We shall latter discuss the procedure to solve the conditions and find $F_p, \theta$.
%
\begin{theorem}~\label{thm:main}
Let us consider a complex matrix $P\in\cnums^{(n+m)\times l}$ and $\epsilon>0$.  The system is $\lambda$-exponentially stable if there exists $s\in\reals^{n}_{\geq 0}$ such that all of the following is true.
\end{theorem}
%
\begin{align*}
    & \cz{\identity{n+m}}{0}{\mymatrix{1 & \dots & 1}^T} \order \cz{P}{0}{s}~\numberthis\label{eqn:incunitbox}\\
    & L = \exp\rb{A_c\rb{\lb+\theta}}P,~~~ H = \mymatrix{\identity{n} ~~~~ 0\\ F_p\exp\rb{-Ac\cdot\theta} }~\numberthis\label{eqn:incbegin}\\
    & \forall i\in\integers_{\geq 0}: \lb+\epsilon\cdot i<\ub, \\
    & \text{If}~ i\cdot\epsilon<\theta,~\text{then}\\
    & M_i = A_r\cdot\exp\rb{A_c\cdot i\cdot\epsilon}~~ \overline{M}_i = M_i\rb{\identity{n+m}+Ac\cdot\epsilon}~\numberthis\\
    & \cz{M_i\errplus \rb{\absolute{M_i}\absolute{A_c}\frac{\epsilon^2}{2}}}{0}{s} \order \cz{\lambda\cdot P}{0}{s}~\numberthis\\
    & \cz{\overline{M}_i\errplus \rb{\absolute{M_i}\absolute{A_c}\frac{\epsilon^2}{2}}}{0}{s} \order \cz{\lambda\cdot P}{0}{s}~\numberthis\label{eqn:incmid1}\\
    & \text{If}~ i\cdot\epsilon\geq\theta,~\text{then}\\
    & M_i = A_r\cdot\exp\rb{A_c\cdot i\cdot\epsilon-\theta}~\numberthis\label{eqn:incmid2}\\
    & \overline{M}_i = M_i\rb{\identity{n+m}+Ac\cdot \epsilon}~\numberthis\\
    & \cz{\rb{M_i\cdot H\cdot P} \errplus \rb{\absolute{M_i}\absolute{A_c}\frac{\epsilon^2}{2}\abs{H\cdot L}}}{0}{s} \\
    & \order \cz{\lambda\cdot P}{0}{s}~\numberthis\\
    & \cz{\rb{\overline{M}_i\cdot H\cdot P} \errplus \rb{\absolute{M_i}\absolute{A_c}\frac{\epsilon^2}{2}\abs{H\cdot L}}}{0}{s}\\
    & \order \cz{\lambda\cdot P}{0}{s}~\numberthis\label{eqn:incend}
\end{align*}
%
\begin{proof}
Let us consider any $t\in\sqb{\lb,\ub}$.  There exist $i\in\integers_{\geq 0}$ such that $i\cdot\epsilon\leq t\leq (i+1)\epsilon$. Using Taylor approximation, we get $\exp\rb{A_c\cdot t}\subseteq $
\[
\exp\rb{A_c\cdot i\cdot\epsilon}\rb{ \rb{ \identity{n+m} + A_c(t-i\cdot \epsilon) } \errplus \absolute{A_c}\cdot \frac{\epsilon^2}{2} }.
\]
%
Then based on Lemma~\ref{lem:convhull} and Equations~\ref{eqn:action1},\ref{eqn:action2}, we get that if $i\cdot\epsilon\geq\theta$, then 

\begin{align*}
\R{t} \subseteq &  \conv{\rb{M_i\cdot H\cdot P}, \rb{\overline{M}_i\cdot H\cdot P}} \\
 & \errplus \rb{\absolute{M_i}\absolute{A_c}\frac{\epsilon^2}{2}\abs{H\cdot L}}.
\end{align*}
Similarly, if $i\cdot\epsilon<\theta$, we get \\ $\R{t}\subseteq \conv{M_i, \overline{M}_i}\errplus \absolute{M_i}\absolute{A_c}\frac{\epsilon^2}{2}$.  Then by Equations~\ref{eqn:incbegin}-\ref{eqn:incend} and Lemma~\ref{lem:linsum}, we get that $\forall t\in\sqb{\lb,\ub}$, $\R{t}\cz{P}{0}{s}\subseteq \lambda\cz{P}{0}{s}$.  Furthermore by~(\ref{eqn:incunitbox}), we get that $\ball{0}{1}\subseteq \cz{P}{0}{s}$.   Therefore, by Lemma~\ref{lem:invstability}, we get that the system is $\lambda$-exponentially stable.
\end{proof}
%
\begin{algorithm}[h!]
\caption{Algorithm to solve for $F_p$ and $\theta$}\label{alg:main}

Choose $\epsilon>0$, $K\in\integers_{\geq 2}$.

Choose $\eta>0$ $~\%$ threshold accuracy of line search.

$\theta_{\min} \gets {\lb}$, $\theta_{\max} \gets \theta_{\min}+\epsilon$, $\theta \gets \theta_{\max}$.

$Feasible\gets~false$.

\While{$\theta_{\max}-\theta_{\min}>\eta$}{
$P\gets E\rb{\theta, K}$.

Using convex optimization search $s$ satisfying (\ref{eqn:incunitbox}-\ref{eqn:incmid1}).

\eIf{(\ref{eqn:incunitbox}-\ref{eqn:incmid1}) are solvable for $s$}{
$Feasible\gets true$.

$s \gets \text{Solution of (\ref{eqn:incunitbox})-(\ref{eqn:incmid1})}$.

$\theta_{\min}\gets \theta$.

}
{
$\theta_{\max}\gets \theta$.
}
$\theta\gets \rb{\theta_{\min} + \theta_{\max}}/{2}$.
}
\eIf{
$Feasible =~true$.
}{
Using convex optimization search minimum norm $F_p$ satisfying (\ref{eqn:incmid2})-(\ref{eqn:incend}).

\eIf{
(\ref{eqn:incmid2})-(\ref{eqn:incend}) are solvable for $F_p$
}{
$F_p\gets \arg \min\set{
\begin{array}{l}
\norm{F_p} \in\reals^{\rb{n+m}\times m} |\\
{ (\ref{eqn:incmid2})-(\ref{eqn:incend})~\text{are true} }
\end{array}
}$.

Return: $F_p,\theta$ (stabilizing feedback matrix acting on previous sample).
}{
Return: solution not found.
}

}{
Return: solution not found.
}
\end{algorithm}
%
The above Theorem~\ref{thm:main} defines sufficient conditions for exponential stability of the system.  For given $\lambda\in[0,1)$, $\lb$, $\ub$, $A_s$, $B_s$, $A_u$, $B_u$ and $F_c$, we can find $F_p$ and $\theta$ as solutions to the above equations.  In particular, we may want to reduce the norm of $F_p$ because larger norm correlates with higher energy.  However, the above constraints can not be solved by straightforward convex optimization because considering all the variables $F_p$, $s$, $P$ and $\theta$, the constraints are non-convex.  Instead, we use the following procedure.

For any $\theta$, we fix the template $P$ of the complex zonotope as the concatenation of column eigenvectors of $R_t$ for different values of $t\in\sqb{\lb,\theta}$.  The reason is, Theorem~\ref{thm:eig} indicates that using eigenvectors of the dynamics as generators can represent contractive complex zonotope.  Henceforth, we sample the eigenvectors of $K+1$ different operators for a chosen $K\in\integers_{\geq 2}$ as follows.
%
\[
P = E\rb{\theta,K} = \mymatrix{\eig\rb{\R{\rb{\lb+i\rb{\theta-\lb}/{K}}}}}_{i=0}^K
\]
%
For this fixed $P$ containing above eigenvectors, we can solve for scale vector $s$ by solving only Equations~\ref{eqn:incunitbox}-\ref{eqn:incmid1} (ignoring rest) using convex optimization (see Remark~\ref{rem:convex}).  However, we need to find a suitable $\theta$.  As $\theta$ is a scalar, we can find the largest possible value of $\theta$ such that Equations~\ref{eqn:incunitbox}-\ref{eqn:incmid1} are solvable for $P =  E\rb{\theta,K}$ using a line search with iterative convex optimization, as described in Algorithm~\ref{alg:main}.   Next, we fix $s$, $\theta$ and $P = E\rb{\theta,K}$ found by the above procedure, and  solve remaining Equations~\ref{eqn:incmid2}-\ref{eqn:incend} and also minimize norm of $F_p$ being searched, using convex optimization.  The overall procedure is described in Algorithm~\ref{alg:main}.
%
