For deriving the sufficient condition to establish stabilization in Theorem~\ref{thm:main}, we use a bound on $\R{t+\delta}$ for $0\leq \delta\leq \epsilon$ given in the following lemma borrowed from ~\cite{hetel2013stabilization}.
%
\begin{lemma}[~\cite{hetel2013stabilization}~]~\label{lem:convhull}
Let $L_0,L_1,...,L_r$ be a finite sequence of real matrices and $U_j\rb{\delta}= \sum_{i=1}^j L_i\delta^i$.  If $0\leq \delta<\epsilon$, then $U_r\rb{\delta} \in \conv{U_{0}\rb{\epsilon},\ldots,U_r\rb{\epsilon}}$.
\end{lemma}
\begin{proof}
This lemma is proved in~\cite{hetel2007lmi}.
\end{proof}
%
The following theorem contains the sufficient conditions for stabilization of the system, which are derived based on the   operations on a complex zonotope described earlier and the above Lemma~\ref{lem:convhull}.  Latter, we will discuss the procedure to solve these constraints and find $F_p, \theta$.
%
\begin{theorem}~\label{thm:main}
Let us consider a complex matrix $P\in\cnums^{(n+m)\times l}$ and $\epsilon>0$.  The system is $\lambda$-exponentially stable if there exists $s\in\reals^{n}_{\geq 0}$ such that all of the following is true.
\end{theorem}
%
\begin{align*}
    & \cz{\identity{n+m}}{0}{\mymatrix{1 & \dots & 1}^T} \order \cz{P}{0}{s}~\numberthis\label{eqn:incunitbox}\\
    & L = \exp\rb{A_c\rb{\lb+\theta}}P~\numberthis\\
    & H = \mymatrix{\identity{n} ~~~~ 0\\ F_p\exp\rb{-A_c\rb{\lb+\theta}} }~\numberthis\label{eqn:incbegin}\\
    & \forall i\in\integers_{\geq 0}: \epsilon\myempty i<\theta, \\
    & M_i = A_r\myempty\exp\rb{A_c\rb{\lb+i\myempty\epsilon}},~ \overline{M}_i = M_i\rb{\identity{n+m}+A_c\myempty\epsilon}~\numberthis\\
    & \cz{M_iP\errplus \rb{\absolute{M_i}\absolute{A_c}\frac{\epsilon^2}{2}\absolute{P}}}{0}{s} \order \cz{\lambda\myempty P}{0}{s}~\numberthis\\
    & \cz{\overline{M}_iP\errplus \rb{\absolute{M_i}\absolute{A_c}\frac{\epsilon^2}{2}\absolute{P}}}{0}{s} \order \cz{\lambda\myempty P}{0}{s}~\numberthis\label{eqn:incmid1}\\
    & \forall i\in\integers_{\geq 0}: \epsilon\myempty i\geq\theta, \\
    & M_i = A_r\myempty\exp\rb{A_c\rb{i\epsilon-\theta}}~\numberthis\label{eqn:incmid2}\\
    & \overline{M}_i = M_i\rb{\identity{n+m}+A_c\myempty \epsilon}~\numberthis\\
    & \cz{\rb{M_i\myempty H\myempty L} \errplus \rb{\absolute{M_i}\absolute{A_c}\frac{\epsilon^2}{2}\abs{H\myempty L}}}{0}{s} \\
    & \order \cz{\lambda\myempty P}{0}{s}~\numberthis\\
    & \cz{\rb{\overline{M}_i\myempty H L} \errplus \rb{\absolute{M_i}\absolute{A_c}\frac{\epsilon^2}{2}\abs{H\myempty L}}}{0}{s}\\
    & \order \cz{\lambda\myempty P}{0}{s}~\numberthis\label{eqn:incend}
\end{align*}
%
\begin{proof}
Let us consider any $t\in\sqb{\lb,\ub}$.  There exist $i\in\integers_{\geq 0}$ such that $\rb{\lb+i\myempty\epsilon}\leq t\leq \rb{\lb+(i+1)\epsilon}$. Using Taylor approximation, we get $\exp\rb{A_c\myempty t}\subseteq $
\[
\exp\rb{A_c \rb{\lb+i\epsilon}}\rb{ \begin{array}{c}
\rb{ \identity{n+m} + A_c\rb{t-\lb-i\myempty \epsilon} }\\
\errplus \rb{\absolute{A_c}\myempty \frac{\epsilon^2}{2}} \end{array}
}.
\]
%
Then based on Lemma~\ref{lem:convhull} and Equations~\ref{eqn:action1},\ref{eqn:action2}, we get that if $i\myempty\epsilon\geq\theta$, then 

\begin{align*}
\R{t}P \subseteq &  \conv{\rb{M_i\myempty H\myempty L}, \rb{\overline{M}_i\myempty H\myempty L}} \\
 & \errplus \rb{\absolute{M_i}\absolute{A_c}\frac{\epsilon^2}{2}\abs{H\myempty L}}.
\end{align*}
Similarly, if $i\myempty\epsilon<\theta$, we get \\ $\R{t}P\subseteq \conv{M_iP, \overline{M}_iP}\errplus \rb{\absolute{M_i}\absolute{A_c}\frac{\epsilon^2}{2}\absolute{P}}$.  Then by Equations~(\ref{eqn:incbegin}-\ref{eqn:incend}) and Lemma~\ref{lem:linsum}, we get that $\forall t\in\sqb{\lb,\ub}$, $\R{t}\real{\cz{P}{0}{s}}\subseteq \lambda\real{\cz{P}{0}{s}}$.  Furthermore by~(\ref{eqn:incunitbox}), we get that $\ball{0}{1}\subseteq \real{\cz{P}{0}{s}}$.   Therefore, by Lemma~\ref{lem:invstability}, taking the $\psi = \real{\cz{P}{c}{s}}$ we get that the system is $\lambda$-exponentially stable.
\end{proof}
%
\begin{algorithm}[h!]
\caption{Algorithm to solve for $F_p$ and $\theta$}\label{alg:main}

Choose $\epsilon>0$, $K\in\integers_{\geq 2}$.

Choose $\eta>0$ $~\%$ threshold accuracy of line search.

Choose $\gamma>0$  $~\%$ bound on norm of $F_p$.

$\theta_{\min} \gets {\lb}$, $\theta_{\max} \gets \ub$, $\theta \gets \rb{\theta_{\max}+\theta_{\min}}/2$.

$Feasible\gets~false$.

\While{$\theta_{\max}-\theta_{\min}>\eta$}{
$P\gets E\rb{\theta, K}$.

Using convex optimization search $s$ satisfying (\ref{eqn:incunitbox}-\ref{eqn:incmid1}) and $\norm{F_p}\leq \gamma$.

\eIf{(\ref{eqn:incunitbox}-\ref{eqn:incmid1}) are solvable for $s$}{
$Feasible\gets true$.

$s \gets \text{Solution of (\ref{eqn:incunitbox})-(\ref{eqn:incmid1})}$.

$\theta_{\min}\gets \theta$.

}
{
$\theta_{\max}\gets \theta$.
}
$\theta\gets \rb{\theta_{\min} + \theta_{\max}}/{2}$.
}
\eIf{
$Feasible =~true$.
}{
Using convex optimization search $F_p$ satisfying (\ref{eqn:incmid2})-(\ref{eqn:incend}) and $\norm{F_p}\leq \gamma$.

\eIf{
(\ref{eqn:incmid2})-(\ref{eqn:incend}) are solvable for $F_p$ and $\norm{F_p}\leq \gamma$
}{
%% $F_p\gets \arg \min\set{
%% \begin{array}{l}
%% \norm{F_p} ~|~ F_p\in\reals^{\rb{n+m}\times m},~ \\
%% { (\ref{eqn:incmid2})-(\ref{eqn:incend})~\text{are true} }
%% \end{array}
%% }$.

Return: $F_p,\theta$ (stabilizing feedback matrix acting on previous sample at $\theta$).
}{
Return: solution not found.
}

}{
Return: solution not found.
}
\end{algorithm}
%
\begin{table*}[t]
$A_s = \mymatrix{ 0 & 1 & 0 & 0 & 0 & 0 & 0 & 0 & 0\\ 0 & 0 &
-1 & 0 & 0 & 0 & 0 & 0 & 0\\ 1.6050 & 4.8680 & -3.5754 & 0 & 0 &
0 & 0 & 0 & 0\\ 0 & 0 & 0 & 0 & 1.0000 & 0 & 0 & 0 & 0\\ 0 & 0 &
1.0000 & 0 & 0 & -1.0000 & 0 & 0 & 0\\ 0 & 0 & 0 & 1.1936 & 3.6258 &
-3.2396 & 0 & 0 & 0\\ 0 & 0 & 0 & 0 & 0 & 0 & 0 & 1.0000 & 0\\ 0 & 0 &
0 & 0 & 0 & 1.0000 & 0 & 0 & -1.0000\\ 0.7132 & 3.5730 & -0.0964 &
0.8472 & 3.2568 & -0.0876 & 1.2726 & 3.0720 & -3.1356 }$,
%
$B_s = \mymatrix{0 & 0\\ 0 & 0\\ 1 & 0\\ 0 & 0\\ 0 & 0\\ 0 & 1\\ 0 &
0\\ 0 & 0\\ 0 & 0}$, $A_u = 0$, $B_u = 0$
\\
$F_c = \mymatrix{0 & 0 & 0 & -0.8198 & 0.4270 & -0.0450 & -0.1942 &
0.3626 v -0.0946\\ 0.8718 & 3.8140 & -0.0754 & 0 & 0 & 0 & -0.5950 &
0.1294 & -0.0796}$.
\caption{Matrices of dynamics for Example~\ref{eg:net}}
\label{tab:mats}
\end{table*}
%
%
\begin{table*}[t]
$F_p = \mymatrix{-0.0194 & 0.0030 & 0.0164 & -0.1402 & -0.0089 & 0.0018 & 0.0090 & 0.0200 & 0.0013 & 0.1230 ~~ -0.00153\\
-0.1052 & 0.0040 & -0.0452 & 0.0225 & -0.0118 & 0.0656 & -0.1270 & 0.0224 & -0.0340 & 0.0190 ~~ -0.0015}$
\caption{Stabilization matrix $F_p$ found for Example~\ref{eg:net}}
\label{tab:res-net}
\end{table*}

The above Theorem~\ref{thm:main} defines sufficient conditions for exponential stability of the system.  For given $\lambda\in[0,1)$, $\lb$, $\ub$, $A_s$, $B_s$, $A_u$, $B_u$ and $F_c$, we can find $F_p$ and $\theta$ as solutions to the above Equations~(\ref{eqn:incunitbox}-\ref{eqn:incend}).  In particular, we may restict the norm of $F_p$ because larger norm correlates with higher energy.  However, the above constraints~(\ref{eqn:incunitbox}-\ref{eqn:incend}) can not be solved by straightforward convex optimization because considering all the variables $F_p$, $s$, $P$ and $\theta$, the constraints are non-convex.  Instead, we use the following procedure.

For any $\theta$, we fix the template $P$ of the complex zonotope as the concatenation of column eigenvectors of $R_t$ for different values of $t\in\sqb{\lb,\theta}$.  The reason is, Theorem~\ref{thm:eig} indicates that using eigenvectors of the dynamics as generators can represent contractive complex zonotope.  Henceforth, we sample the eigenvectors of $K+1$ different operators for a chosen $K\in\integers_{\geq 2}$ as follows.
%
\[
P = E\rb{\theta,K} = \mymatrix{\eig\rb{\R{\rb{\lb+i\rb{\theta-\lb}/{K}}}}}_{i=0}^K
\]
%
We also choose a small value for the $\epsilon>0$.   For this fixed $P$ containing above eigenvectors and $\epsilon>0$, we can solve for scale vector $s$ by solving only Equations~(\ref{eqn:incunitbox}-\ref{eqn:incmid1}) (ignoring rest) using convex optimization (see Remark~\ref{rem:convex}).  Still, we need to find a suitable $\theta$.  As $\theta$ is a scalar, we can find the largest possible value of $\theta$ such that Equations~(\ref{eqn:incunitbox}-\ref{eqn:incmid1}) are solvable for $P =  E\rb{\theta,K}$ using a line search with iterative convex optimization, as described in Algorithm~\ref{alg:main}.   Next, we fix the $s$, $\theta$ and $P = E\rb{\theta,K}$ that are found by the above procedure, and  solve remaining Equations~(\ref{eqn:incmid2}-\ref{eqn:incend}), while also bounding the norm of $F_p$ being searched, using convex optimization.  The overall procedure is described in Algorithm~\ref{alg:main}.
%
