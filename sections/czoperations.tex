A simple zonotope is a Minkowski sum of line segments in the real euclidean space described as linear combination of real-valued vectors, called generators, with bounded combining coefficients.  However, the eigenstructure of the dynamics, which is complex valued, is closely related to finding invariants or contractive sets (see below Theorem~\ref{thm:eig}).  Since real zonotopes can not incorporate complex valued eigenvectors as generators, we extended real zonotopes to complex zonotopes in~\cite{adimoolam2016using,adimoolam2016template,adimoolam2018calculus}.  A complex zonotope is a linear combination of complex-valued vectors with complex combining coefficients bounded inside circles of complex plane.  Complex zonotopes are geometrically more expressive as they can be Minkowski sum of some embedded ellipses in addition to line segments.   They are defined below.
%
\begin{definition}
Let ${ P\in\cnums^{\rows{P}\times\cols{P}} }$ be a complex matrix, ${c\in\reals^{\rows{P}}}$ be a real vector and $s\in\reals^{\cols{P}}_{\geq 0}$ be a non-negative real vector. The following is a complex zonotope centered at $c$ with template $P$ and scale vector $s$.%
\begin{align*}
    & \cz{P}{c}{s} = \set{P\zeta + c\st{\zeta\in\cnums^{\cols{P}},~\abs{\zeta}\leq s}}
\end{align*}
%
\end{definition}
%
The following theorem illustrates the relation between eigenstructure of the system and finding contractive complex zonotopes.
%
\begin{theorem}~\label{thm:eig}
Let a complex matrix $P$ contain the eigenvectors of $\R{t}$ for some $t$.  If the eigenvalues of $\R{t}$ are all less than $\lambda$, then $\R{t}\cz{P}{0}{s}\subseteq \lambda\cz{P}{0}{s}$ for all $s\in\reals_{\geq 0}$.
\end{theorem}
%
\begin{proof}
The theorem is proved in~\cite{adimoolam2018calculus} (Lemma~5.4.1).
\end{proof}
%
The linear transformation of a complex zonotope gives another complex zonotope, which can also be computed very efficiently as follows.
%
\begin{lemma}~\label{lem:linsum}~\cite{adimoolam2018calculus}
Let us consider a complex zonotope $\cz{P}{c}{s}$ and a real matrix $A$.  Then the following is true.
%
\begin{align*}
    & A\cz{P}{c}{s} = 
    \cz{AP}{Ac}{s}.~\numberthis\label{eqn:minsum}
\end{align*}
%
\end{lemma}
%
\begin{proof}
The lemma is proved in~\cite{adimoolam2018calculus}.
\end{proof}
%
The algorithm we propose later to stabilize the system involves checking inclusion of a set of complex zonotopes, obtained by applying a sequence of transformations, inside another original complex zonotope (see Lemma~\ref{lem:invstability}).  In this regard, we define a set of complex zonotopes whose templates lie in the neighborhood of a given template as follows.  Let us consider a real matrix with positive entries $\Upsilon\in\reals_{\geq 0}^{\rows{Q}\times\cols{Q}}$ where $Q$ is a complex matrix and a real vector $\rho\in\reals^{\cols{e}}$ where $e$ is a real vector.  
%
\begin{align*}
    & \cz{Q\errplus\Upsilon}{e\errplus\rho}{r} \\
    & = \set{\cz{Q+\widehat{Q}}{e+u}{r}~\st{\abs{Q}\leq \Upsilon,~\abs{u}\leq \rho}}
\end{align*}
%
The following relation is a sufficient condition for checking the required inclusion which is proved later in Lemma~\ref{lem:inclusion}.  The following Lemma~\ref{lem:inclusion} is an extension of the inclusion checking between two complex zonotopes proved in~\cite{adimoolam2018calculus} (Theorem 2.3.8 ).
%
\begin{definition}~\label{defn:inc}  Let $P$ be a complex matrix such that $P^T P$ is a square invertible matrix.  Let $\Upsilon>0$ be a real matrix with only positive elements.  We define the relation \[ \cz{Q\errplus \Upsilon}{e\errplus\rho}{r} \order \cz{P}{c}{s}\] if all of the following conditions are verified:
%
\begin{align*}
    & \exists X,\Delta\in\cnums^{\cols{P}\times\cols{Q}}, y\in\cnums^{\cols{P}}:\\
    & PX = Q\diag{r},~~ \Delta = \abs{\pinv{P}}\Upsilon\diag{r},~\numberthis\label{eqn:inc1}\\
    & \left(e-c\right) = Py,~~\delta = \abs{\pinv{P}}{\rho},\numberthis\\
    & \forall i\in\set{1,\ldots,\rows{X}}\\
    & \abs{y_i} + \delta_i + \sum_{j=1}^{\cols{X}}\abs{X_{ij}} + \Delta_{ij}\leq s_i~\numberthis\label{eqn:inc2}
\end{align*}
\end{definition}
%
\begin{lemma}~\label{lem:inclusion}
If $ \cz{Q\errplus\Upsilon}{e\errplus\rho}{r}\order \cz{P}{c}{s}$ is true for $\Upsilon,\rho>0$, then the subset inclusion $ \cz{Q\errplus\Upsilon}{e\errplus\rho}{r}\subseteq \cz{P}{c}{s}$ is true. 
\end{lemma}
%
\begin{proof}
Let us assume that  $ \cz{Q\errplus \Upsilon}{e\errplus \rho}{r}\order \cz{P}{c}{s}$ is true.  Hence, there exist matrices $X,\Delta$ and complex vectors $y,\rho$ such that all the equations in~(\ref{eqn:inc2}) are true.
%
% \begin{align*}
% & PX = Q\diag{r},~~\left(e-c\right) = Py\\
% & \forall i\in\set{1,\ldots,\rows{X}} ~~\abs{y_i} + \sum_{j=1}^{\cols{X}}\abs{X_{ij}} \leq s_i.~\numberthis\label{proof:inclusion}
% \end{align*}
%
Let us consider any $x\in\cz{Q\errplus \Upsilon}{e}{R}$.  Based on the definition of a complex zonotope, there exists $\zeta\in\cnums^{\cols{P}}$ such that $\abs{\zeta}\leq s$ and $x = \rb{Q+\widehat{Q}}\zeta + e + u$, $\abs{\widehat{Q}}\leq \Upsilon$ and $\abs{u}\leq \rho$.  We now have to show that $x\in \cz{P}{c}{s}$.  

Let us consider a vector $\alpha\in\cnums^{\cols{\zeta}}$  such that for any $i\in\set{1,\ldots,\cols{\zeta}}$, the following is true:
%
\begin{align*}
    & \alpha_i = \zeta_i/r_i ~~\text{if}~~r_i>0,
    & \alpha_i = 0~~\text{if}~~r_i = 0~\numberthis
\end{align*}
%
Since $\abs{\zeta}\leq r$, it follows from the above definition that $\abs{\alpha}\leq 1$ and  $\zeta = \diag{r}\alpha$. 
Then we derive the following.
%
\begin{align*}
    & x = \rb{Q+\widehat{Q}}\zeta + e + u = \rb{Q+\widehat{Q}}\diag{r}\alpha + e + u \\
    & = \rb{Q+\widehat{Q}}\diag{r}\alpha + (e-c) + c + u
\end{align*}
%
and from \eqref{eqn:inc1}
\begin{align*}
    & x = P\left(X\alpha + \pinv{P}\widehat{Q}\diag{r}\alpha + y + \pinv{P}u\right) + c~\numberthis\label{proof:inc:exp}
\end{align*}
We derive the following for any $i\in\set{1,\ldots, \rows{X}}$
%
\begin{align*}
 &   \abs{X\alpha + \pinv{P}\widehat{Q}\diag{r}\alpha + y +\pinv{P}u}_i\\
 & \leq \abs{y}_i + \abs{\pinv{P}}\abs{u}  \\
 & ~~~~+ \sum_{j=1}^{\cols{X}} \rb{\abs{X}_{ij}+\rb{\abs{\pinv{P}}\abs{\widehat{Q}}\diag{r}}_{ij}}\abs{\alpha}_j\\
 & \leq \abs{y}_i + \delta_i + \sum_{j=1}^{\cols{X}}\abs{X}_{ij}+\Delta_{ij}\\
     & \leq s_i.~\numberthis\label{proof:inc:bound}
\end{align*}
%
The above second inequality is true because $\abs{\alpha}\leq 1$, $\abs{\widehat{Q}}\leq \Upsilon$, $\Delta = \abs{\pinv{P}}\Upsilon\diag{r}$, $\abs{u}\leq\rho$ and $\pinv{P}\rho = \delta$, and the last inequality is deduced from \eqref{eqn:inc2}.
%
From \eqref{proof:inc:exp} and \eqref{proof:inc:bound}, we get that $x\in\cz{P}{c}{s}$.  As this is true for any $x\in\cz{Q}{e}{r}$, the inclusion $\cz{Q}{e}{r}\subseteq \cz{P}{c}{s}$ is true.  
\end{proof}
%
\begin{remark}~\label{rem:convex}
For fixed $P$, $Q$ and $\Upsilon$, the Equations~\ref{eqn:inc1}-\ref{eqn:inc2} define convex constraints on $s$.  In fact, the inequalities can be recast as a set of second order conic constraints on $s$ and the remaining variables.  
Similarly, for fixed $s$, the equations define convex constraints on $P$, $Q$, and $\Upsilon$.
\end{remark}
