%%%%%%%%%%%%%%%%%%%%%%%%%%%%%%%%%%%%%%%%%%%%%%%%%%%%%%%%%%%%%%%%%%%%%%%%%%%%%%%%
%2345678901234567890123456789012345678901234567890123456789012345678901234567890
%        1         2         3         4         5         6         7         8

\documentclass[letterpaper, 10 pt, conference]{ieeeconf}  % Comment this line out
                                                          % if you need a4paper
%\documentclass[a4paper, 10pt, conference]{ieeeconf}      % Use this line for a4
                                                          % paper

\IEEEoverridecommandlockouts                              % This command is only
                                                          % needed if you want to
                                                          % use the \thanks command
\overrideIEEEmargins
% See the \addtolength command later in the file to balance the column lengths
% on the last page of the document



% The following packages can be found on http:\\www.ctan.org
%\usepackage{graphics} % for pdf, bitmapped graphics files
%\usepackage{epsfig} % for postscript graphics files
%\usepackage{mathptmx} % assumes new font selection scheme installed
%\usepackage{times} % assumes new font selection scheme installed
%\usepackage{amsmath} % assumes amsmath package installed
%\usepackage{amssymb}  % assumes amsmath package installed

\usepackage{cite}
\usepackage[utf8]{inputenc}
\usepackage{graphicx}
\usepackage{amsmath}
\usepackage{amsfonts}
\usepackage{color}
\usepackage{amssymb}
\usepackage{mathtools}
\usepackage{float}
\usepackage{multirow}
\usepackage{caption}
\usepackage{algorithm2e}

\newcommand\numberthis{\addtocounter{equation}{1}\tag{\theequation}} % numbering equations
\RestyleAlgo{boxruled}

\bibliographystyle{plain}
\input{macros}

\title{\LARGE \bf
Synthesizing Stabilization Feedback using Complex Zonotope
}

%\author{ \parbox{3 in}{\centering Huibert Kwakernaak*
%         \thanks{*Use the $\backslash$thanks command to put information here}\\
%         Faculty of Electrical Engineering, Mathematics and Computer Science\\
%         University of Twente\\
%         7500 AE Enschede, The Netherlands\\
%         {\tt\small h.kwakernaak@autsubmit.com}}
%         \hspace*{ 0.5 in}
%         \parbox{3 in}{ \centering Pradeep Misra**
%         \thanks{**The footnote marks may be inserted manually}\\
%        Department of Electrical Engineering \\
%         Wright State University\\
%         Dayton, OH 45435, USA\\
%         {\tt\small pmisra@cs.wright.edu}}
%}

\author{Arvind Adimoolam and Thao Dang% <-this % stops a space
\thanks{}% write supporting organization
\thanks{Thao Dang is with the CNRS, and the Verimag Laboratory, Universit{\'e} Grenoble Alpes, France
        {\tt\small thao.dang@univ-grenoble-alpes.fr}}%
\thanks{Arvind Adimoolam is with the Department of Computer Science and Engineering, IIT Kanpur,
        India.
        {\tt\small santosh@cse.iitk.ac.in}}%
}


\begin{document}



\maketitle
\thispagestyle{empty}
\pagestyle{empty}


%%%%%%%%%%%%%%%%%%%%%%%%%%%%%%%%%%%%%%%%%%%%%%%%%%%%%%%%%%%%%%%%%%%%%%%%%%%%%%%%
\begin{abstract}
We consider the problem of stabilizing a linear sampled data control
system based on information from previous samples, when the current
sample information is unavailable and also the sampling time is
uncertain.  The control structure can account for distributed
components where theres is possible communication loss resulting in
unavailability of state information in sufficient time.  Therefore, we
focus on synthesizing a stabilizing feedback based on previous sample
to increases the robustness of system to uncertainty in sampling time.
However, simply using time evolution to map the feedback based on
current sample to the feedback based on previous sample may not
provide stabilization for large enough uncertainty in sampling period.
Alternatively, we propose a more effective stabilization technique that
employs complex zonotopes generated by eigenvectors of the dynamics to
represent invariant sets, from which the feedback matrix acting on the
previous state sample is derived.  Complex zonotopes are an extension
of real-valued zonotopes to the complex number domain, which has been
previously used to effectively verify stability of sampled data
control systems.  In this paper, we extend the application of complex
zonotopes to control synthesis.  We evaluate our approach on benchmark
examples and demonstrate stabilization on larger ranges of uncertainty
on sampling time periods than the uncertainty ranges proposed in the
benchmarks.
\end{abstract}
%%%%%%%%%%%%%%%%%%%%%%%%%%%%%%%%%%%%%%%%%%%%%%%%%%%%%%%%%%%%%%%%%%%%%%%%%%%%%%%%

{\keywords Controller Synthesis, Sampled Data, Information loss, Complex Zonotope.}

\section{Introduction}~\label{sec:intro}
Sampled data systems are systems where the feedback input is periodically sampled based on information received about the state of the system, typically over a communication network.  They can be used to control distributed components of a system via communication networks.  But communication networks are prone to loss of information, which can result
in subsequent inactuation and instability of a sampled data system.  In the case of information loss,
we may expect to stabilize the system by applying feedback based
on information from previous samples.  However, the sampling time can
be uncertain and time varying in realistic systems, due to which the
problem of finding appropriate stabilizing feedback based on the
previous sample becomes challenging.  

For instance, the expected
feedback at the current time instant can be written as a function of
the previous sample (see Remark~\ref{rem:counterintuitive}).  But
counterintuitively, such feedback may not effectively stabilize the
system since the time at which the future sample information becomes
available is uncertain (\emph{this is explained in
  Example~\ref{eg:counterintuitive}}).  There can be other
feedbacks acting on previous samples which may stabilize the system more robustly, i.e., for
larger ranges of uncertainty in the sampling time.  Earlier research
works~\cite{briat2013convex,hetel2013stabilization} have tackled the
problem of finding stabilizing feedback based on the current sample
when the sampling time is uncertain.  However, to our knowledge, this
problem of stabilizing system based on previous sample information
when the current sample information is missing and the future sampling
time in uncertain, has not been tacked yet.

In this paper, we come up with an algorithm to find the
stabilizing feedback based on previous sample when the current sample
information is missing and the sampling time is uncertain.  To this
end, we find a novel application of a set representation called
\emph{complex zonotope}~\cite{adimoolam2018calculus}.  Complex
zonotope is an extension of real valued zonotopes to complex plane
which has been earlier applied to verify stability of sampled data
systems based on its complex valued
eigenstructure~\cite{adimoolam2016using,adimoolam2017template}.  In
this paper, we extend the application of complex zonotope to
synthesize stabilizing feedback acting on previous sample, given range
of uncertainty in sampling time.  We make use of the possibly complex
valued eigenstructure of the dynamics to find an candidate complex
zonotopic invariant for the system and then synthesize the required
stabilizing feedback acting on the previous sample.


%
\subsection{Related Work}
Before continuing we briefly discuss related work. The uncertainty induced  communication networks and protocols in NCS, such as  variable sampling period, variable communication delays, and packet loss, has been intensively considered in the literature which has become very vast. We therefore refer the reader to a thorough survey~\cite{Heemels2010}. Concerning continuous-time models as used in our work, delay-differential equations (DDEs) have been used to model variable sampling intervals and their stability is studied using Lyapunov-Krasovskii functional method and linear matrix inequality-based techniques. Additionally Impulsive DDEs are also used to capture better the update nature of control laws.  When not all the data is used, the uncertain delays can be captured in one delay variable that determines the most recent available control input. Then, constructing polytopic embeddings of the uncertain system are proposed using interval matrices, the real Jordan form, the Taylor series. Concerning stabilizing feedback controller design, our approach is similar to the one using dynamical control law with a lifted state variable. However, instead of using LMI techniques we used set-based stability conditions and complex zonotopes to find stabilizing feedback controllers. Since complex zonotopes allow capturing eigenstructure of the dynamics, stability conditions can be checked accurately, leading to larger solution sets.
%
\subsection{Notation}
We denote real numbers by $\reals$ and complex numbers by $\cnums$.  For a real number subset $S\subseteq \reals$, $r\in\reals$ and $\bowtie\in\set{\leq, <,\geq,>}$, we denote $S_{\bowtie r} = \set{x\in S\st{x\bowtie r}}$.  If $X,Y\in\reals^{n\times m}$ are two real matrices of same size, then we say $X\bowtie Y$ if $\forall i,j~X_{ij}\bowtie Y_ij$.  Similarly, for a scalar $r\in\reals$, $X\bowtie r$ if $\forall i,j~X_{ij}\bowtie r$.  For a complex number $z = x+\iota y$, $\abs{z} = \sqrt{x^2+y^2}$.  For a complex matrix $X\in\cnums^{n\times m}$, $\abs{X}$ is a real matrix where $\forall i,j~\abs{X}_{ij} = \abs{X_{ij}}$.  The euclidean norm of a complex vector $x\in\cnums^n$ is $\norm{x}$ and that of a complex matrix $X\in\cnums^{n\times m}$ is $\norm{X}$.  The real valued ball of radius $r$ around a real vector $x\in\reals^n$ is $\ball{x}{r} = \set{y\in\reals^n\st{\norm{x-y}}\leq r}$.  The real projection of a set $S\subseteq \cnums^n$ is $\real{S}$.


\section{Problem}~\label{sec:problem}
We consider a sampled data linear time invariant (LTI) system  defined as follows. The state of the system at time $t\in[0,\infty)$ is $\state{t}\in\reals^n$, the feedback input  is $\inp{t}\in\reals^m$, $\lb$ and $\ub$ are lower and upper bounds, respectively, on feedback sampling period, $A_s\in\reals^{n\times n}$, $B_s\in\reals^{n\times m}$, $A_u\in\reals^{m\times n}$, $B_u\in\reals^{m\times m}$, $F_p\in\reals^{m\times (n+m)}$ and $F_c\in\reals^{m\times n}$ are real matrices related to the system dynamics described below. The matrix $F_c$ acts on the current sample and $F_p$ acts on previous sample and input.  The value $\theta$ is the time elapse after $\lb$ at which $F_p$ is applied on the previous sample to reset the feedback input.  But we apply the feedback  at most once between successive state sampling.  Allowing multiple such historical feedbacks between successive state samples makes the stabilization problem more difficult, which is beyond the scope of this paper.  Henceforth, we take $\theta\geq\rb{\ub+\lb}/2$ so that the feedback based on previous sample is applied at most once between successive state sampling.
%
\begin{align*}
    & \theta \geq \rb{\ub+\lb}/2\\
    & \exists \rb{t_k}_{k=0}^\infty,~\forall k\in\integers_{\geq 0} ~t_k\in\reals_{\geq 0}\\
    & \forall t\notin\bigcup_{k=0}^\infty\rb{ \set{t_k}\bigcup \set{t_k+\theta\st{ t_k+\theta< t_{k+1} } } },\\
    &\der{\state{t}} = A_s\state{t} + B_s{\inp{t}},~~\der{{\inp{t}}} = A_u\state{t} + B_u{\inp{t}}~\numberthis\label{eqn:ncsstart}\\
    & \forall k\in\integers_{\geq 0},
     \rb{t_{k+1}-t_k}\in[\lb,\ub],~t_0 = 0,\\
    & \inp{t_k} = F_c\state{t_k}^-~\numberthis\\
    &~\text{If}~\theta<\rb{t_{k+1}-t_k},~\text{then}~\inp{\rb{t_k+\theta}} = F_p\mymatrix{\state{t_k}^-\\\inp{t_k}^-}~\numberthis\label{eqn:ncsend}.
\end{align*}
%
We denote the combined vector of the state of the system and controller input, called {\em combined state}, at any time $t$ as $\cs{t} = \mymatrix{\state{t}^T,\inp{t}^T}^T$.  We call a sequence of combined states at sampling times $\rb{\cs{t_k}}_{k=0}^\infty$ a sampling time trajectory which, by simple manipulation of the above equations, can be shown to be equivalently governed by the following dynamics:
%
\begin{align*}
    & A_r = \mymatrix{\identity{n} & 0\\ F_c & 0},~~H = \mymatrix{\identity{n} ~~~~ 0\\ F_p\exp\rb{-A_c\rb{\theta+\lb}} },\\
    & A_c = \mymatrix{A_s & B_s \\ A_u & B_u}.~\numberthis\label{eqn:impulse}\\
    & \cs{t_{k+1}} = \R{\rb{t_{k+1}-t_k}}\cs{t_k}~~\text{where}\\
    & \forall \tau\in[\lb,\lb+\theta],~\R{\tau} = A_r\exp\rb{A_c\tau}.~\numberthis\label{eqn:action1}\\
    & \forall \tau\in(\lb+\theta,\ub],~\text{as}~\theta\geq\rb{\ub+\lb}/2, \\
    & \R{\tau} = A_r\exp\rb{A_c\rb{\tau-\lb-\theta}}H\exp\rb{A_c\rb{\lb+\theta}}.~\numberthis\label{eqn:action2}
\end{align*}
%
\begin{definition}[Exponential stability]
The system is said to be $\lambda$-exponentially stable for $\lambda\in[0,1)$ if there exists $r\in\reals_{\geq 0}$ such that for every sequence of sampling times $\rb{t_k}_{k=0}^\infty$, the following is true.
%
\begin{align*}
    & \norm{\cs{t_k}}\leq r\lambda^k\norm{\cs{0}} ~\numberthis\label{eqn:expstability}
\end{align*}
%
\end{definition}
%
\begin{problem}
We are given $\lambda\in[0,1)$, $\lb$, $\ub$, $A_s$, $B_s$, $A_u$, $B_u$ and $F_c$ such that the matrix $\R{\lb}$ is Schur.  Find a feedback matrix  $F_p\in\reals^{m\times\rb{n+m}}$ acting on the previous state sample and input and also find a $\theta\in\sqb{\rb{\ub+\lb}/2, \ub}$ such that the system is $\lambda$-exponentially stable.
\end{problem}
%
\begin{remark}~\label{rem:counterintuitive}  Let us consider that a system is stable for sampling times between $\sqb{\lb,\lb+\mu}$ without the previous sample based feedback.  The expected sampling time feedback at a time $\rb{t_k+\mu}$ is ${F_c\state{\rb{t_k+\mu}}}$.  As a function of the previous sample $\state{t_k}$, this feedback can be written as $F_c\exp\rb{A_c\rb{\lb+\mu}}\state{t_k}$. Therefore, when the sample information is missing at $\rb{t_k+\lb+\mu}$, it may seem intuitive to apply feedback $F_c\exp\rb{A_c\rb{\lb+\mu}}\state{t_k}$ at $\rb{\lb+\mu}$ to stabilize the system for a larger $\ub>\lb+\mu$.   However, counterintuitively, this may not provide stabilization for a large enough $\ub$ but other values of $F_p$ may lead to far more robust stabilization, as illustrated in the below example.  
\end{remark}
%
\begin{example}~\label{eg:counterintuitive}
Let consider the benchmarked sampled data system taken from~\cite{hetel2013stabilization,al2015stability}, where $A_s = \mymatrix{0 & 1\\ -2 & 0.1}$, $B = \mymatrix{0 \\ 1}$, $A_u = \mymatrix{0 & 0}$, $B_u = 0$ and $F_c = \mymatrix{1 & 0}$.  
%
Without previous sample based feedback,~\cite{al2015stability} has
shown this system is stable for sampling times in $\sqb{0.4,1.888}$,
but unstable for $\ub\geq 1.889$.  If we set a matching feedback
matrix $F_p = F_c\exp\rb{1.888A_c}$ acting on previous sample at
sampling time $1.888$ where $\theta = 1.488$, the system will be
unstable for $\ub\geq 1.898$, which is a very small increment $(0.01)$ than the previous stability threshold $1.888$.  This is because $\R{0.4}\R{1.898} =
A_r\exp\rb{0.4A_c}A_r\exp\rb{0.01 A_c}A_r\exp\rb{1.888 A_c}$ has
eigenvalues outside the unit complex circle.  Instead, if we take $F_p
= \mymatrix{0.1422 & 0.4669 & 0.1417}$ with $\theta = 0.41$, we can
stabilize the system for a far larger value $\ub = 3.42$, which we
shall deduce later in this paper.  This shows that the matching time
evolution based feedback $F_p = F_c\exp\rb{A_c\rb{\lb+\mu}}$ at some
$\rb{\lb+\mu}$ need not be an efficient solution.  Moreover, there is
no guarantee that such feedback will stabilize the system for a larger
$\ub>\lb+\mu$.
\end{example}

Alternatively, we shall describe a stabilization method using complex zonotopes, which stabilizes the system for large uncertainties in the sampling period on benchmark examples.


\section{Solution}~\label{sec:solution}
%\subsection{Relation between invariant set and stability}
The stability of the system can be established by the existence of a contractive set around origin containing the ball of unit radius, as described in the following lemma.
%
\begin{lemma}~\label{lem:invstability}
The system is $\lambda$-exponentially stable if there exists a set $\Psi\subseteq\reals$ such that $\max_{z\in\Psi}\norm{z}<\infty$, $\ball{0}{1}\subseteq \Psi$ and $\forall \tau\in\sqb{\lb,\ub}$ $\R{\tau}\Psi\subseteq \lambda\Psi$.
\end{lemma}
%
\begin{proof}
Let us denote $r = \max_{z\in\Psi}\norm{z}$ which is finite as considered in the above lemma.  We have $\cs{0}\subseteq \norm{\cs{0}}\ball{0}{1}\subseteq\norm{\cs{0}}\Psi$.  Then we derive the following.
%
\begin{align*}
    & \cs{k} = \prod_{i=1}^k\R{\rb{t_k-t_{k-1}}}\cs{i} \subseteq \prod_{i=1}^k\R{\rb{t_k-t_{k-1}}}\norm{\cs{0}}\Psi\\
    & = \norm{\cs{0}}\prod_{i=1}^k\R{\rb{t_k-t_{k-1}}}\Psi \subseteq \norm{\cs{0}}\lambda^k\Psi
\end{align*}
%
Then we get $\norm{\cs{k}}\leq \norm{\cs{0}}\lambda^k\max_{z\in\Psi}\norm{z} = r\norm{\cs{0}}\lambda^k$.  As this $r$ is constant for all initial states $\cs{0}$, we get that the system is $\lambda$-exponentially stable.
\end{proof}
%
Therefore, to stabilize the system, we can find an $F_p$, $\theta$ and a set $\Psi$ satisfying the conditions of Lemma~\ref{lem:invstability}.  The choice of the set $\Psi$ affects the robustness of our solution, i.e., the $\ub$ for which stabilization can be achieved.  It has been previously shown that using the eigenstructure of the system dynamics inside a set representation called complex zonotope~\cite{adimoolam2016using}, stability can be verified for larger ranges of $\ub$ on benchmark examples than other set methods.  So, in this paper, our choice for the $\Psi$ is a complex zonotope and we extend its application to stabilize the system, i.e. find $F_p$ and $\theta$.  Firstly, we discuss necessary operations on a complex zonotope in the following section.
%
\subsection{Complex Zonotope and Set Operations}
A simple zonotope is a Minkowski sum of line segments in the real euclidean space described as linear combination of real-valued vectors, called generators, with bounded combining coefficients.  However, the eigenstructure of the dynamics, which is complex valued, is closely related to finding invariants or contractive sets (see below Theorem~\ref{thm:eig}).  Since real zonotopes can not incorporate complex valued eigenvectors as generators, we extended real zonotopes to complex zonotopes in~\cite{adimoolam2016using,adimoolam2016template,adimoolam2018calculus}.  A complex zonotope is a linear combination of complex-valued vectors with complex combining coefficients bounded inside circles of complex plane.  Complex zonotopes are geometrically more expressive as they can be Minkowski sum of some embedded ellipses in addition to line segments.   They are defined below.
%
\begin{definition}
Let ${ P\in\cnums^{\rows{P}\times\cols{P}} }$ be a complex matrix, ${c\in\reals^{\rows{P}}}$ be a real vector and $s\in\reals^{\cols{P}}_{\geq 0}$ be a non-negative real vector. The following is a complex zonotope centered at $c$ with template $P$ and scale vector $s$.%
\begin{align*}
    & \cz{P}{c}{s} = \set{P\zeta + c\st{\zeta\in\cnums^{\cols{P}},~\abs{\zeta}\leq s}}
\end{align*}
%
\end{definition}
%
The following theorem illustrates the relation between eigenstructure of the system and finding contractive complex zonotopes.
%
\begin{theorem}~\label{thm:eig}
Let a complex matrix $P$ contain the eigenvectors of $\R{t}$ for some $t$.  If the eigenvalues of $\R{t}$ are all less than $\lambda$, then $\R{t}\cz{P}{0}{s}\subseteq \lambda\cz{P}{0}{s}$ for all $s\in\reals_{\geq 0}$.
\end{theorem}
%
\begin{proof}
The theorem is proved in~\cite{adimoolam2018calculus} (Lemma~5.4.1).
\end{proof}
%
The linear transformation of a complex zonotope gives another complex zonotope, which can also be computed very efficiently as follows.
%
\begin{lemma}~\label{lem:linsum}~\cite{adimoolam2018calculus}
Let us consider a complex zonotope $\cz{P}{c}{s}$ and a real matrix $A$.  Then the following is true.
%
\begin{align*}
    & A\cz{P}{c}{s} = 
    \cz{AP}{Ac}{s}.~\numberthis\label{eqn:minsum}
\end{align*}
%
\end{lemma}
%
\begin{proof}
The lemma is proved in~\cite{adimoolam2018calculus}.
\end{proof}
%
The algorithm we propose later to stabilize the system involves checking inclusion of a set of complex zonotopes, obtained by applying a sequence of transformations, inside another original complex zonotope (see Lemma~\ref{lem:invstability}).  In this regard, we define a set of complex zonotopes whose templates lie in the neighborhood of a given template as follows.  Let us consider a real matrix with positive entries $\Upsilon\in\reals_{\geq 0}^{\rows{Q}\times\cols{Q}}$ where $Q$ is a complex matrix and a real vector $\rho\in\reals^{\cols{e}}$ where $e$ is a real vector.  
%
\begin{align*}
    & \cz{Q\errplus\Upsilon}{e\errplus\rho}{r} \\
    & = \set{\cz{Q+\widehat{Q}}{e+u}{r}~\st{\abs{Q}\leq \Upsilon,~\abs{u}\leq \rho}}
\end{align*}
%
The following relation is a sufficient condition for checking the required inclusion which is proved later in Lemma~\ref{lem:inclusion}.  The following Lemma~\ref{lem:inclusion} is an extension of the inclusion checking between two complex zonotopes proved in~\cite{adimoolam2018calculus} (Theorem 2.3.8 ).
%
\begin{definition}~\label{defn:inc}  Let $P$ be a complex matrix such that $P^T P$ is a square invertible matrix.  Let $\Upsilon>0$ be a real matrix with only positive elements.  We define the relation \[ \cz{Q\errplus \Upsilon}{e\errplus\rho}{r} \order \cz{P}{c}{s}\] if all of the following conditions are verified:
%
\begin{align*}
    & \exists X,\Delta\in\cnums^{\cols{P}\times\cols{Q}}, y\in\cnums^{\cols{P}}:\\
    & PX = Q\diag{r},~~ \Delta = \abs{\pinv{P}}\Upsilon\diag{r},~\numberthis\label{eqn:inc1}\\
    & \left(e-c\right) = Py,~~\delta = \abs{\pinv{P}}{\rho},\numberthis\\
    & \forall i\in\set{1,\ldots,\rows{X}}\\
    & \abs{y_i} + \delta_i + \sum_{j=1}^{\cols{X}}\abs{X_{ij}} + \Delta_{ij}\leq s_i~\numberthis\label{eqn:inc2}
\end{align*}
\end{definition}
%
\begin{lemma}~\label{lem:inclusion}
If $ \cz{Q\errplus\Upsilon}{e\errplus\rho}{r}\order \cz{P}{c}{s}$ is true for $\Upsilon,\rho>0$, then the subset inclusion $ \cz{Q\errplus\Upsilon}{e\errplus\rho}{r}\subseteq \cz{P}{c}{s}$ is true. 
\end{lemma}
%
\begin{proof}
Let us assume that  $ \cz{Q\errplus \Upsilon}{e\errplus \rho}{r}\order \cz{P}{c}{s}$ is true.  Hence, there exist matrices $X,\Delta$ and complex vectors $y,\rho$ such that all the equations in~(\ref{eqn:inc2}) are true.
%
% \begin{align*}
% & PX = Q\diag{r},~~\left(e-c\right) = Py\\
% & \forall i\in\set{1,\ldots,\rows{X}} ~~\abs{y_i} + \sum_{j=1}^{\cols{X}}\abs{X_{ij}} \leq s_i.~\numberthis\label{proof:inclusion}
% \end{align*}
%
Let us consider any $x\in\cz{Q\errplus \Upsilon}{e}{R}$.  Based on the definition of a complex zonotope, there exists $\zeta\in\cnums^{\cols{P}}$ such that $\abs{\zeta}\leq s$ and $x = \rb{Q+\widehat{Q}}\zeta + e + u$, $\abs{\widehat{Q}}\leq \Upsilon$ and $\abs{u}\leq \rho$.  We now have to show that $x\in \cz{P}{c}{s}$.  

Let us consider a vector $\alpha\in\cnums^{\cols{\zeta}}$  such that for any $i\in\set{1,\ldots,\cols{\zeta}}$, the following is true:
%
\begin{align*}
    & \alpha_i = \zeta_i/r_i ~~\text{if}~~r_i>0,
    & \alpha_i = 0~~\text{if}~~r_i = 0~\numberthis
\end{align*}
%
Since $\abs{\zeta}\leq r$, it follows from the above definition that $\abs{\alpha}\leq 1$ and  $\zeta = \diag{r}\alpha$. 
Then we derive the following.
%
\begin{align*}
    & x = \rb{Q+\widehat{Q}}\zeta + e + u = \rb{Q+\widehat{Q}}\diag{r}\alpha + e + u \\
    & = \rb{Q+\widehat{Q}}\diag{r}\alpha + (e-c) + c + u
\end{align*}
%
and from \eqref{eqn:inc1}
\begin{align*}
    & x = P\left(X\alpha + \pinv{P}\widehat{Q}\diag{r}\alpha + y + \pinv{P}u\right) + c~\numberthis\label{proof:inc:exp}
\end{align*}
We derive the following for any $i\in\set{1,\ldots, \rows{X}}$
%
\begin{align*}
 &   \abs{X\alpha + \pinv{P}\widehat{Q}\diag{r}\alpha + y +\pinv{P}u}_i\\
 & \leq \abs{y}_i + \abs{\pinv{P}}\abs{u}  \\
 & ~~~~+ \sum_{j=1}^{\cols{X}} \rb{\abs{X}_{ij}+\rb{\abs{\pinv{P}}\abs{\widehat{Q}}\diag{r}}_{ij}}\abs{\alpha}_j\\
 & \leq \abs{y}_i + \delta_i + \sum_{j=1}^{\cols{X}}\abs{X}_{ij}+\Delta_{ij}\\
     & \leq s_i.~\numberthis\label{proof:inc:bound}
\end{align*}
%
The above second inequality is true because $\abs{\alpha}\leq 1$, $\abs{\widehat{Q}}\leq \Upsilon$, $\Delta = \abs{\pinv{P}}\Upsilon\diag{r}$, $\abs{u}\leq\rho$ and $\pinv{P}\rho = \delta$, and the last inequality is deduced from \eqref{eqn:inc2}.
%
From \eqref{proof:inc:exp} and \eqref{proof:inc:bound}, we get that $x\in\cz{P}{c}{s}$.  As this is true for any $x\in\cz{Q}{e}{r}$, the inclusion $\cz{Q}{e}{r}\subseteq \cz{P}{c}{s}$ is true.  
\end{proof}
%
\begin{remark}~\label{rem:convex}
For fixed $P$, $Q$ and $\Upsilon$, the Equations~\ref{eqn:inc1}-\ref{eqn:inc2} define convex constraints on $s$.  In fact, the inequalities can be recast as a set of second order conic constraints on $s$ and the remaining variables.  
Similarly, for fixed $s$, the equations define convex constraints on $P$, $Q$, and $\Upsilon$.
\end{remark}

%
\subsection{Stabilization using complex zonotope}
For deriving the sufficient condition to establish stabilization in Theorem~\ref{thm:main}, we use a bound on $\R{t+\delta}$ for $0\leq \delta\leq \epsilon$ given in the following lemma borrowed from ~\cite{hetel2013stabilization}.
%
\begin{lemma}[~\cite{hetel2013stabilization}~]~\label{lem:convhull}
Let $L_0,L_1,...,L_r$ be a finite sequence of real matrices and $U_j\rb{\delta}= \sum_{i=1}^j L_i\delta^i$.  If $0\leq \delta<\epsilon$, then $U_r\rb{\delta} \in \conv{U_{0}\rb{\epsilon},\ldots,U_r\rb{\epsilon}}$.
\end{lemma}
\begin{proof}
This lemma is proved in~\cite{hetel2007lmi}.
\end{proof}
%
The following theorem contains the sufficient conditions for stabilization of the system, which are derived based on the   operations on a complex zonotope described earlier and the above Lemma~\ref{lem:convhull}.  Latter, we will discuss the procedure to solve these constraints and find $F_p, \theta$.
%
\begin{theorem}~\label{thm:main}
Let us consider a complex matrix $P\in\cnums^{(n+m)\times l}$ and $\epsilon>0$.  The system is $\lambda$-exponentially stable if there exists $s\in\reals^{n}_{\geq 0}$ such that all of the following is true.
\end{theorem}
%
\begin{align*}
    & \cz{\identity{n+m}}{0}{\mymatrix{1 & \dots & 1}^T} \order \cz{P}{0}{s}~\numberthis\label{eqn:incunitbox}\\
    & L = \exp\rb{A_c\rb{\lb+\theta}}P~\numberthis\\
    & H = \mymatrix{\identity{n} ~~~~ 0\\ F_p\exp\rb{-A_c\rb{\lb+\theta}} }~\numberthis\label{eqn:incbegin}\\
    & \forall i\in\integers_{\geq 0}: \epsilon\myempty i<\theta, \\
    & M_i = A_r\myempty\exp\rb{A_c\rb{\lb+i\myempty\epsilon}},~ \overline{M}_i = M_i\rb{\identity{n+m}+A_c\myempty\epsilon}~\numberthis\\
    & \cz{M_iP\errplus \rb{\absolute{M_i}\absolute{A_c}\frac{\epsilon^2}{2}\absolute{P}}}{0}{s} \order \cz{\lambda\myempty P}{0}{s}~\numberthis\\
    & \cz{\overline{M}_iP\errplus \rb{\absolute{M_i}\absolute{A_c}\frac{\epsilon^2}{2}\absolute{P}}}{0}{s} \order \cz{\lambda\myempty P}{0}{s}~\numberthis\label{eqn:incmid1}\\
    & \forall i\in\integers_{\geq 0}: \epsilon\myempty i\geq\theta, \\
    & M_i = A_r\myempty\exp\rb{A_c\rb{i\epsilon-\theta}}~\numberthis\label{eqn:incmid2}\\
    & \overline{M}_i = M_i\rb{\identity{n+m}+A_c\myempty \epsilon}~\numberthis\\
    & \cz{\rb{M_i\myempty H\myempty L} \errplus \rb{\absolute{M_i}\absolute{A_c}\frac{\epsilon^2}{2}\abs{H\myempty L}}}{0}{s} \\
    & \order \cz{\lambda\myempty P}{0}{s}~\numberthis\\
    & \cz{\rb{\overline{M}_i\myempty H L} \errplus \rb{\absolute{M_i}\absolute{A_c}\frac{\epsilon^2}{2}\abs{H\myempty L}}}{0}{s}\\
    & \order \cz{\lambda\myempty P}{0}{s}~\numberthis\label{eqn:incend}
\end{align*}
%
\begin{proof}
Let us consider any $t\in\sqb{\lb,\ub}$.  There exist $i\in\integers_{\geq 0}$ such that $\rb{\lb+i\myempty\epsilon}\leq t\leq \rb{\lb+(i+1)\epsilon}$. Using Taylor approximation, we get $\exp\rb{A_c\myempty t}\subseteq $
\[
\exp\rb{A_c \rb{\lb+i\epsilon}}\rb{ \begin{array}{c}
\rb{ \identity{n+m} + A_c\rb{t-\lb-i\myempty \epsilon} }\\
\errplus \rb{\absolute{A_c}\myempty \frac{\epsilon^2}{2}} \end{array}
}.
\]
%
Then based on Lemma~\ref{lem:convhull} and Equations~\ref{eqn:action1},\ref{eqn:action2}, we get that if $i\myempty\epsilon\geq\theta$, then 

\begin{align*}
\R{t}P \subseteq &  \conv{\rb{M_i\myempty H\myempty L}, \rb{\overline{M}_i\myempty H\myempty L}} \\
 & \errplus \rb{\absolute{M_i}\absolute{A_c}\frac{\epsilon^2}{2}\abs{H\myempty L}}.
\end{align*}
Similarly, if $i\myempty\epsilon<\theta$, we get \\ $\R{t}P\subseteq \conv{M_iP, \overline{M}_iP}\errplus \rb{\absolute{M_i}\absolute{A_c}\frac{\epsilon^2}{2}\absolute{P}}$.  Then by Equations~(\ref{eqn:incbegin}-\ref{eqn:incend}) and Lemma~\ref{lem:linsum}, we get that $\forall t\in\sqb{\lb,\ub}$, $\R{t}\real{\cz{P}{0}{s}}\subseteq \lambda\real{\cz{P}{0}{s}}$.  Furthermore by~(\ref{eqn:incunitbox}), we get that $\ball{0}{1}\subseteq \real{\cz{P}{0}{s}}$.   Therefore, by Lemma~\ref{lem:invstability}, taking the $\psi = \real{\cz{P}{c}{s}}$ we get that the system is $\lambda$-exponentially stable.
\end{proof}
%
\begin{algorithm}[h!]
\caption{Algorithm to solve for $F_p$ and $\theta$}\label{alg:main}

Choose $\epsilon>0$, $K\in\integers_{\geq 2}$.

Choose $\eta>0$ $~\%$ threshold accuracy of line search.

Choose $\gamma>0$  $~\%$ bound on norm of $F_p$.

$\theta_{\min} \gets {\lb}$, $\theta_{\max} \gets \ub$, $\theta \gets \rb{\theta_{\max}+\theta_{\min}}/2$.

$Feasible\gets~false$.

\While{$\theta_{\max}-\theta_{\min}>\eta$}{
$P\gets E\rb{\theta, K}$.

Using convex optimization search $s$ satisfying (\ref{eqn:incunitbox}-\ref{eqn:incmid1}) and $\norm{F_p}\leq \gamma$.

\eIf{(\ref{eqn:incunitbox}-\ref{eqn:incmid1}) are solvable for $s$}{
$Feasible\gets true$.

$s \gets \text{Solution of (\ref{eqn:incunitbox})-(\ref{eqn:incmid1})}$.

$\theta_{\min}\gets \theta$.

}
{
$\theta_{\max}\gets \theta$.
}
$\theta\gets \rb{\theta_{\min} + \theta_{\max}}/{2}$.
}
\eIf{
$Feasible =~true$.
}{
Using convex optimization search $F_p$ satisfying (\ref{eqn:incmid2})-(\ref{eqn:incend}) and $\norm{F_p}\leq \gamma$.

\eIf{
(\ref{eqn:incmid2})-(\ref{eqn:incend}) are solvable for $F_p$ and $\norm{F_p}\leq \gamma$
}{
%% $F_p\gets \arg \min\set{
%% \begin{array}{l}
%% \norm{F_p} ~|~ F_p\in\reals^{\rb{n+m}\times m},~ \\
%% { (\ref{eqn:incmid2})-(\ref{eqn:incend})~\text{are true} }
%% \end{array}
%% }$.

Return: $F_p,\theta$ (stabilizing feedback matrix acting on previous sample at $\theta$).
}{
Return: solution not found.
}

}{
Return: solution not found.
}
\end{algorithm}
%
\begin{table*}[t]
$A_s = \mymatrix{ 0 & 1 & 0 & 0 & 0 & 0 & 0 & 0 & 0\\ 0 & 0 &
-1 & 0 & 0 & 0 & 0 & 0 & 0\\ 1.6050 & 4.8680 & -3.5754 & 0 & 0 &
0 & 0 & 0 & 0\\ 0 & 0 & 0 & 0 & 1.0000 & 0 & 0 & 0 & 0\\ 0 & 0 &
1.0000 & 0 & 0 & -1.0000 & 0 & 0 & 0\\ 0 & 0 & 0 & 1.1936 & 3.6258 &
-3.2396 & 0 & 0 & 0\\ 0 & 0 & 0 & 0 & 0 & 0 & 0 & 1.0000 & 0\\ 0 & 0 &
0 & 0 & 0 & 1.0000 & 0 & 0 & -1.0000\\ 0.7132 & 3.5730 & -0.0964 &
0.8472 & 3.2568 & -0.0876 & 1.2726 & 3.0720 & -3.1356 }$,
%
$B_s = \mymatrix{0 & 0\\ 0 & 0\\ 1 & 0\\ 0 & 0\\ 0 & 0\\ 0 & 1\\ 0 &
0\\ 0 & 0\\ 0 & 0}$, $A_u = 0$, $B_u = 0$
\\
$F_c = \mymatrix{0 & 0 & 0 & -0.8198 & 0.4270 & -0.0450 & -0.1942 &
0.3626 v -0.0946\\ 0.8718 & 3.8140 & -0.0754 & 0 & 0 & 0 & -0.5950 &
0.1294 & -0.0796}$.
\caption{Matrices of dynamics for Example~\ref{eg:net}}
\label{tab:mats}
\end{table*}
%
%
\begin{table*}[t]
$F_p = \mymatrix{-0.0194 & 0.0030 & 0.0164 & -0.1402 & -0.0089 & 0.0018 & 0.0090 & 0.0200 & 0.0013 & 0.1230 ~~ -0.00153\\
-0.1052 & 0.0040 & -0.0452 & 0.0225 & -0.0118 & 0.0656 & -0.1270 & 0.0224 & -0.0340 & 0.0190 ~~ -0.0015}$
\caption{Stabilization matrix $F_p$ found for Example~\ref{eg:net}}
\label{tab:res-net}
\end{table*}

The above Theorem~\ref{thm:main} defines sufficient conditions for exponential stability of the system.  For given $\lambda\in[0,1)$, $\lb$, $\ub$, $A_s$, $B_s$, $A_u$, $B_u$ and $F_c$, we can find $F_p$ and $\theta$ as solutions to the above Equations~(\ref{eqn:incunitbox}-\ref{eqn:incend}).  In particular, we may restict the norm of $F_p$ because larger norm correlates with higher energy.  However, the above constraints~(\ref{eqn:incunitbox}-\ref{eqn:incend}) can not be solved by straightforward convex optimization because considering all the variables $F_p$, $s$, $P$ and $\theta$, the constraints are non-convex.  Instead, we use the following procedure.

For any $\theta$, we fix the template $P$ of the complex zonotope as the concatenation of column eigenvectors of $R_t$ for different values of $t\in\sqb{\lb,\theta}$.  The reason is, Theorem~\ref{thm:eig} indicates that using eigenvectors of the dynamics as generators can represent contractive complex zonotope.  Henceforth, we sample the eigenvectors of $K+1$ different operators for a chosen $K\in\integers_{\geq 2}$ as follows.
%
\[
P = E\rb{\theta,K} = \mymatrix{\eig\rb{\R{\rb{\lb+i\rb{\theta-\lb}/{K}}}}}_{i=0}^K
\]
%
We also choose a small value for the $\epsilon>0$.   For this fixed $P$ containing above eigenvectors and $\epsilon>0$, we can solve for scale vector $s$ by solving only Equations~(\ref{eqn:incunitbox}-\ref{eqn:incmid1}) (ignoring rest) using convex optimization (see Remark~\ref{rem:convex}).  Still, we need to find a suitable $\theta$.  As $\theta$ is a scalar, we can find the largest possible value of $\theta$ such that Equations~(\ref{eqn:incunitbox}-\ref{eqn:incmid1}) are solvable for $P =  E\rb{\theta,K}$ using a line search with iterative convex optimization, as described in Algorithm~\ref{alg:main}.   Next, we fix the $s$, $\theta$ and $P = E\rb{\theta,K}$ that are found by the above procedure, and  solve remaining Equations~(\ref{eqn:incmid2}-\ref{eqn:incend}), while also bounding the norm of $F_p$ being searched, using convex optimization.  The overall procedure is described in Algorithm~\ref{alg:main}.
%



\section{Evaluation}~\label{sec:evaluation}
For convex optimization in our algorithm, we used CVX~\cite{grant2008cvx}, a package for specifying and solving convex programs.
\begin{example}
% We consider a sampled data system taken from~\cite{hetel2013stabilization}, where $A_s = \mymatrix{0 & 1\\ -2 & 0.1}$, $B = \mymatrix{0 \\ 1}$, $A_u = \mymatrix{0 & 0}$, $B_u = 0$ and $F_c = \mymatrix{1 & 0}$.  
Let us consider the system described in Example~\ref{eg:counterintuitive} in Section~\ref{sec:problem}.
\end{example}
Without past sample based feedback,~\cite{al2015stability} has shown this system is stable for sampling times in $\sqb{0.4,1.888}$, but unstable for $\ub\geq 1.889$.  We want to exponentially stabilize the system for a large $\ub>1.889$, given $\lb = 0.4$ and $\lambda = 0.9999$, using previous sample based feedback at appropriate sampling period $\theta$.  However, if we set a matching feedback $F_p = F_s\exp\rb{A_c\cdot 1.888}$ at $\theta = 1.888$, the system is unstable for $\ub\geq 2.088$.  This is because $\R{2.088}\cdot\R{0.4} = A_r\exp\rb{0.4 A_c}A_r\exp\rb{0.2 A_c}A_r\exp\rb{1.888 A_c}$ has eigenvalues outside unit complex circle.  Therefore, we can not increase the sampling time range of stability by a large value by using a matching feedback input as above.  On the other hand, our algorithm does increase the sampling time range of stability by a large value $\ub = 3.42$ as given in the following result.

\emph{Settings}. We took $K = 2$, $\eta = 0.01$ and $\epsilon = 0.01$.

\emph{Result}.  For $\ub = 3.42$, our algorithm found $F_p = \mymatrix{0.1422 &   0.4669  &  0.1417}$ at $\theta = 1.81$ to $\lambda$-exponentially stabilize the system.  

\begin{example}
We consider a sampled data system taken from~\cite{hetel2017recent}, where $A_s = \mymatrix{1 & 3\\ 2 & 1}$, $B = \mymatrix{1 \\ 0.6}$, $A_u = \mymatrix{0 & 0}$, $B_u = 0$ and $F_c = \mymatrix{-1 & -6}$.  
\end{example}
Without past sample based feedback, given $\tau_{min} = 0.18$, the system is shown in~\cite{hetel2017recent} to be unstable for $\tau_{max}\geq 0.54$.  Given $\lambda = 0.999$, we want to $\lambda$-exponentially stabilize the system for a large $\tau_{max}> 0.54$. 

\emph{Settings}. We took $K = 2$, $\eta = 0.01$ and $\epsilon = 0.01$.

\emph{Result}.   For $\ub = 0.8$, our algorithm found $F_p = \mymatrix{-10.5819  &  1.7823 &  -2.5906}$ at $\theta = 0.49$ to exponentially stabilize the system.  

\begin{example}
We consider a sampled data system taken from~\cite{hetel2013stabilization}, where $A_s = \mymatrix{0 & -3\\ 1.4 & -2.6}$, $B = \mymatrix{1 \\ 0.6}$, $A_u = \mymatrix{8.4 -18.6}$, $B_u = 4.6$ and $F_c = \mymatrix{0 & 0}$.  
\end{example}
Without past sample based feedback, given $\tau_{min} = 0.1$, the system is shown in~\cite{al2015stability} to be unstable for $\tau_{max}\geq 0.515$.  Given $\lambda = 0.99$, we want to $\lambda$-exponentially stabilize the system for a large $\tau_{max}> 0.515$ by finding a suitable $F_p$ and $\theta$.

\emph{Settings}. We took $K = 2$, $\eta = 0.001$ and $\epsilon = 0.001$.

\emph{Result}.   For $\ub = 0.86$, our algorithm found $F_p = {10^{-8}\myempty\mymatrix{0.1667 &  -0.0259 &   0.0561}}$ at $\theta = 0.48$ to exponentially stabilize the system.


\section{Conclusion}
Since sampled data systems are prone to information loss that can affect stability, we can expect to more robustly stabilize the system by applying feedback based on the previous sample.  However, finding an appropriate feedback based on previous sample and the time at which to apply such feedback is challenging because of uncertainty in the time at which the future state information becomes available.  We explained this in the paper by an example.  To synthesize effective feedback based on previous sample, we developed a novel algorithm based on complex zonotopes which use eigenstructure of the dynamics.  Our evaluation on benchmark examples demonstrates that the algorithm can stabilize the system for much larger values of uncertainty in sampling time than the values for which it was originally stable.  

\todo{please include future work}

\bibliography{ref}



\end{document}
